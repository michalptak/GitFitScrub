\documentclass[a4paper]{article}
\usepackage{polski}
\usepackage[utf8]{inputenc}
\usepackage{url}
\usepackage{navigator}
\embeddedfile{main}{./main.tex}

\title{\bf{Aplikacja mobilna dla biegacza - GitFitScrub}}
\author{{\em Michał Ptak, Katarzyna Poręba, Jan Michalik}}
\date{}

\begin{document}

\begin{titlepage}
\maketitle
\thispagestyle{empty}
\bigskip
\begin{center}
Zespołowe przedsięwzięcie inżynierskie\\[2mm]

Informatyka\\[2mm]

Rok. akad. 2017/2018, sem. I\\[2mm]

Prowadzący: dr hab. Marcin Mazur
\end{center}
\end{titlepage}

\tableofcontents
\thispagestyle{empty}

\newpage

\section{Opis projektu}

\subsection{Członkowie zespołu}

\begin{enumerate}
\item Michał Ptak (kierownik projektu).
\item Katarzyna Poręba.
\item Jan Michalik.
\end{enumerate}

\subsection{Cel projektu (produkt)}

Naszym celem jest zbudowanie aplikacji mobilnej na system operacyjny Android przy użyciu map Google, która pobiera informacje poruszającej się osoby z modułu GPS i zapisuje poczynania w bazie danych, a także gromadzi statystyki dotyczące długości trasy, prędkości czy spalanych kalorii.

\subsection{Potencjalny odbiorca produktu (klient)}

% Określić potencjalnego klienta (wraz z uzasadnieniem).
Klient biega na różnych odcinkach miasta. Oczekuje w pełni darmowej aplikacji bez reklam, która będzie śledzić trasę po której się porusza. Potrzebuje również statystyk swoich postępów.

\subsection{Metodyka}

Projekt będzie realizowany przy użyciu (zaadaptowanej do istniejących warunków) metodyki {\em Scrum}.

\section{Wymagania użytkownika}
<<Przedstawić listę wymagań użytkownika w postaci ,,historyjek'' (User stories). Każda historyjka powinna opisywać jedną cechę systemu. Struktura: As a [type of user], I want [to perform some task] so that I can [achieve some goal/benefit/value] (zob. np. \cite{us}).>>

\subsection{User story 1}
Jako stały użytkownik, aktywnie spędzający czas, chcę mieć dostęp do mapy, aby widzieć jaką trasą się poruszam.

\subsection{User story 2}
<<Historyjka 2>>.

\subsection*{<<Tutaj dodawać kolejne historyjki>>}

\section{Harmonogram}

\subsection{Rejestr zadań (Product Backlog)}

\begin{itemize}
\item Data rozpoczęcia: <<data>>.
\item  Data zakończenia: <<data>>.
\end{itemize}

\subsection{Sprint 1}

\begin{itemize}
\item Data rozpoczęcia: <<data>>.
\item Data zakończenia: <<data>>.
\item Scrum Master: <<imię i nazwisko>>.
\item Product Owner: <<imię i nazwisko>>.
\item Development Team: <<lista developerów>>.
\end{itemize}

\subsection{Sprint 2}

\begin{itemize}
\item Data rozpoczęcia: <<data>>.
\item  Data zakończenia: <<data>>.
\item Scrum Master: <<imię i nazwisko>>.
\item Product Owner: <<imię i nazwisko>>.
\item Development Team: <<lista developerów>>.
\end{itemize}

\subsection*{<<Tutaj dodawać kolejne Sprint'y>>}

\section{Product Backlog}

\subsection{Backlog Item 1}
\paragraph{Tytuł zadania.} <<Tytuł>>.
\paragraph{Opis zadania.} <<Opis>>.
\paragraph{Priorytet.} <<Priorytet>>.
\paragraph{Definition of Done.} <<Określić (w języku zrozumiałym dla wszystkich członków zespołu), co oznacza ukończenie danego zadania>>.

\subsection{Backlog Item 2}
\paragraph{Tytuł zadania.} <<Tytuł>>.
\paragraph{Opis zadania.} <<Opis>>.
\paragraph{Priorytet.} <<Priorytet>>.
\paragraph{Definition of Done.} <<Określić (w języku zrozumiałym dla wszystkich członków zespołu), co oznacza ukończenie danego zadania>>.

\subsection*{<<Tutaj dodawać kolejne zadania>>}

\section{Sprint 1}
\subsection{Cel} <<Określić, w jakim celu tworzony jest przyrost produktu>>.
\subsection{Sprint Planning/Backlog}

\paragraph{Tytuł zadania.} <<Tytuł>>.
\begin{itemize}
\item Estymata: <<szacowana czasochłonność (w ,,koszulkach'')>>.
\end{itemize}

\paragraph{Tytuł zadania.} <<Tytuł>>.
\begin{itemize}
\item Estymata: <<szacowana czasochłonność (w ,,koszulkach'')>>.
\end{itemize}

\paragraph{<<Tutaj dodawać kolejne zadania>>}

\subsection{Realizacja} % dodać narzędzia z których skorzystaliśmy

\paragraph{Tytuł zadania.} <<Tytuł>>.
\subparagraph{Wykonawca.} <<Wykonawca>>.
\subparagraph{Realizacja.} <<Sprawozdanie z realizacji zadania (w tym ocena zgodności z estymatą). Kod programu (środowisko \texttt{verbatim}): \begin{verbatim}
for (i=1; i<10; i++)
...
\end{verbatim}>>.

\paragraph{Tytuł zadania.} <<Tytuł>>.
\subparagraph{Wykonawca.} <<Wykonawca>>.
\subparagraph{Realizacja.} <<Sprawozdanie z realizacji zadania (w tym ocena zgodności z estymatą). Kod programu (środowisko \texttt{verbatim}): \begin{verbatim}
for (i=1; i<10; i++)
...
\end{verbatim}>>.

\paragraph{<<Tutaj dodawać kolejne zadania>>}


\subsection{Sprint Review/Demo}
<<Sprawozdanie z przeglądu Sprint'u -- czy założony cel (przyrost) został osiągnięty oraz czy wszystkie zaplanowane Backlog Item'y zostały zrealizowane? Demostracja przyrostu produktu>>.

\section{Sprint 2}

\subsection{Cel} <<Określić, w jakim celu tworzony jest przyrost produktu>>.

\subsection{Sprint Planning/Backlog}

\paragraph{Tytuł zadania.} <<Tytuł>>.
\begin{itemize}
\item Estymata: <<szacowana czasochłonność (w ,,koszulkach'')>>.
\end{itemize}

\paragraph{Tytuł zadania.} <<Tytuł>>.
\begin{itemize}
\item Estymata: <<szacowana czasochłonność (w ,,koszulkach'')>>.
\end{itemize}

\paragraph{<<Tutaj dodawać kolejne zadania>>}

\subsection{Realizacja}

\paragraph{Tytuł zadania.} <<Tytuł>>.
\subparagraph{Wykonawca.} <<Wykonawca>>.
\subparagraph{Realizacja.} <<Sprawozdanie z realizacji zadania (w tym ocena zgodności z estymatą). Kod programu (środowisko \texttt{verbatim}): \begin{verbatim}
for (i=1; i<10; i++)
...
\end{verbatim}>>.

\paragraph{Tytuł zadania.} <<Tytuł>>.
\subparagraph{Wykonawca.} <<Wykonawca>>.
\subparagraph{Realizacja.} <<Sprawozdanie z realizacji zadania (w tym ocena zgodności z estymatą). Kod programu (środowisko \texttt{verbatim}): \begin{verbatim}
for (i=1; i<10; i++)
...
\end{verbatim}>>.

\paragraph{<<Tutaj dodawać kolejne zadania>>}


\subsection{Sprint Review/Demo}
<<Sprawozdanie z przeglądu Sprint'u -- czy założony cel (przyrost) został osiągnięty oraz czy wszystkie zaplanowane Backlog Item'y zostały zrealizowane? Demostracja przyrostu produktu>>.

\section*{<<Tutaj dodawać kolejne Sprint'y>>}


\begin{thebibliography}{9}

\bibitem{Cov} S. R. Covey, {\em 7 nawyków skutecznego działania}, Rebis, Poznań, 2007.

\bibitem{csharp} Ian Griffiths, {\em C\# 5.0. Programowanie. Tworzenie aplikacji Windows 8, internetowych oraz biurowych w .NET 4.5 Framework}, O’Reilly, 2013.

\bibitem{Oet} Tobias Oetiker i wsp., Nie za krótkie wprowadzenie do systemu \LaTeX  \ $2_\varepsilon$, \url{ftp://ftp.gust.org.pl/TeX/info/lshort/polish/lshort2e.pdf}

\bibitem{SchSut} K. Schwaber, J. Sutherland, {\em Scrum Guide}, \url{http://www.scrumguides.org/}, 2016.

\bibitem{apr} \url{https://agilepainrelief.com/notesfromatooluser/tag/scrum-by-example}

\bibitem{us} \url{https://www.tutorialspoint.com/scrum/scrum_user_stories.htm}

\end{thebibliography}

\end{document}

% ----------------------------------------------------------------
